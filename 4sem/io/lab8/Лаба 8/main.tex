\documentclass{report}


\usepackage{cmap}


\pdfgentounicode=1 % Для корректного отображения текста при копировании
\usepackage{embedall} % Принудительно встраивает все шрифты

\usepackage[russian]{babel}
\usepackage{geometry}
\usepackage{minted2}
\usepackage[hidelinks]{hyperref}
\usepackage{xcolor}
\usepackage{xspace}
\usepackage{fancyhdr}
\usepackage{tocloft}
\usepackage{booktabs}
\usepackage{tikz}
\usepackage{titlesec}
\usepackage{tabularx}
\usepackage{setspace}
\usepackage{makecell}
\usepackage{graphicx}
\usepackage{caption}
\usepackage{amsmath}
\usepackage{array}
\usepackage{cancel}
\usepackage{multirow}
\usepackage{amsmath}
\usepackage{amssymb}

\usetikzlibrary{shapes, arrows}
\usetikzlibrary{positioning}
\usetikzlibrary{shapes.geometric}
\usetikzlibrary{shapes.misc}
\usetikzlibrary{calc}
\usetikzlibrary{chains}
\usetikzlibrary{babel}

\tikzstyle{pf} = [circle, draw, text centered, minimum height = 2em]
\tikzstyle{rec} = [rectangle, draw, text centered, rounded corners, minimum height=2em]
\tikzstyle{proc} = [rectangle, draw, text centered, minimum height=2em]
\tikzstyle{dec} = [diamond, draw, text centered, minimum height=2em]
\tikzstyle{io}=[trapezium, draw, text centered, trapezium left angle=60, trapezium right angle=120, minimum height=2em]
\tikzstyle{cyc}=[chamfered rectangle, draw, text centered,chamfered rectangle xsep=10pt]
\tikzstyle{con} = [draw, -latex']

\tikzset{
	up left connect/.style={
		to path={(\tikztostart.south) -- ++ (0, -0.15) -- ++ (-#1, 0) |- (\tikztotarget)}
	},
	down right connect/.style={
		to path={(\tikztostart) -- ++ (#1, 0) |- ($(\tikztotarget.north) + (0, 0.2)$) -- (\tikztotarget)}
	}
}

\newcommand{\drawzalupa}[2]{
	\begin{figure}[H]
		%\begin{center}
		\centering
		\includegraphics[width=#2\textwidth]{#1}
		%\end{center}
	\end{figure}
}

\setlength{\cftaftertoctitleskip}{2pt}

\renewcommand{\cfttoctitlefont}{\LARGE\bfseries}

\titleformat{\chapter}{\normalfont\LARGE\bfseries}{\thechapter}{20pt}{\LARGE\bfseries}
\titlespacing*{\chapter}{0pt}{0pt}{20pt}

\pagestyle{fancy}

\fancyhf{}

\fancyfoot[C]{\thepage}

\renewcommand{\headrulewidth}{0pt}

\renewcommand{\footrulewidth}{0pt}


\definecolor{pybg}{rgb}{0.95,0.95,0.95}


\geometry{left=3cm}
\geometry{right=1.5cm}
\geometry{top=2cm}
\geometry{bottom=2cm}

\counterwithout{figure}{chapter}

\newcommand{\q}[1]{``#1''}

\newminted[code]{Python}{
	bgcolor=pybg,
	baselinestretch=1.2,
	fontsize=\normalsize,
	tabsize=0,
	linenos,
	obeytabs=true,
	tabsize=4
}

\begin{document}
	\begin{titlepage}
		\begin{center}
			МИНИСТЕРСТВО НАУКИ И ВЫСШЕГО ОБРАЗОВАНИЯ\linebreak 
			РОССИЙСКОЙ ФЕДЕРАЦИИ\medskip
			
			ФЕДЕРАЛЬНОЕ ГОСУДАРСТВЕННОЕ БЮДЖЕТНОЕ ОБРАЗОВАТЕЛЬНОЕ 
			УЧРЕЖДЕНИЕ ВЫСШЕГО ОБРАЗОВАНИЯ\medskip
			
			\textbf{
				«БЕЛГОРОДСКИЙ ГОСУДАРСТВЕННЫЙ \linebreak
				ТЕХНОЛОГИЧЕСКИЙ УНИВЕРСИТЕТ им. В. Г. ШУХОВА»\linebreak
				(БГТУ им. В.Г. Шухова)
			}\bigskip
			
			Кафедра программного обеспечения вычислительной техники и автоматизированных систем
			\vspace{5cm}
			
			\Large\textbf{Лабораторная работа №8}
			
			\large по дисциплине: Исследование операций\linebreak
			тема: \q{Задачи дробно-линейного программирования}
		\end{center}\vspace{6cm}
		
		\begin{flushright}
			\begin{minipage}{7cm}
				Выполнил: ст. группы ПВ-231\\
				Столяров Захар\\
				\\
				Проверил: \\
				Вирченко Юрий Петрович\\
			\end{minipage}
		\end{flushright}\bigskip
		
		\
		
		\
		
		\
		
		\
		
		
		\begin{center}
			Белгород, 2025 г.
		\end{center}
	\end{titlepage}
	
	\newpage
	
	\setcounter{secnumdepth}{-1}
	\chapter{Лабораторная работа №8 <<Задачи дробно-линейного программирования (задачи ДЛП)>>}
	
	\textbf{Цель работы:} Освоить метод сведения задачи ДЛП к задаче
	линейного программирования с помощью введения новых переменных.
	Изучить алгоритм решения задачи ДЛП и реализовать программно
	этот алгоритм.
	
	\begin{center}
		\textbf{Вариант 14}
	\end{center}
	\begin{center}
		\textbf{Задания для подготовки к работе}
	\end{center}
	
	\begin{enumerate}
		\item{Изучить постановку задачи ДЛП, а также подходы к ее решению.} 
		\item{Ознакомиться с введением новых переменных, в которых задача
			ДЛП превращается в задачу ЛП.} 
		\item{Изучить метод и алгоритм решения задачи ДЛП, составить и
			отладить программу решения этой задачи, в качестве
			тестовых данных решив аналитически следующую задачу:}
	\end{enumerate}
	
	
	\begin{center}
		{
		\begin{equation*}
			z = \frac{9x_1 + 3x_2 + 2x_3 + x_4}{2x_1 + x_2 + 3x_3 + x_4} \rightarrow \max;
		\end{equation*}
		
		\begin{equation*}
			\begin{cases}
				6x_1 + x_2 + x_3 + 3x_4 \leq 280, \\
				x_1 + 3x_3 + 2x_4 \leq 70, \\
				2x_1 + 4x_2 + x_3 \leq 320, \\
				
			\end{cases}
		\end{equation*}
		
		\begin{equation*}
			x_i \geq 0 \, (i = 1, 4)
		\end{equation*}
		}
	\end{center}
	
	
	
	\newpage
	
	\chapter{Программное решение}
	
	\section{Блок-схемы основных функций программы:}
	
	\begin{center}
		\begin{tikzpicture}[node distance=1.5cm]
			\node (func) [rec] {compute\_quotient(vars)};
			\node (num) [proc, below=of func] 
			{Вычислить числитель:  $9y_1 + 3y_2 + 2y_3 + y_4$};
			\node (denom) [proc, below=of num] 
			{Вычислить знаменатель:  $2y_1 + y_2 + 3y_3 + y_4$};
			\node (ret) [proc, below=of denom] 
			{Вернуть $-numer/denom$};
			
			\draw[con] (func) -- (num);
			\draw[con] (num) -- (denom);
			\draw[con] (denom) -- (ret);
		\end{tikzpicture}
	\end{center}
	
	% Блок-схема 2: optimize_fractional_problem()
	\newpage
	\begin{center}
		\begin{tikzpicture}[node distance=1.5cm]
			\node (func) [rec] {optimize\_fractional\_problem()};
			\node (init) [proc, below=of func] 
			{Инициализация границ и начального приближения};
			\node (constr) [proc, below=of init] 
			{Определение ограничений:  
				1. $6x_1 + x_2 + x_3 + 3x_4 \leq 280$ 
				2. $x_1 + 3x_3 + 2x_4 \leq 70$ 
				3. $2x_1 + 4x_2 + x_3 \leq 320$};
			\node (solve) [proc, below=of constr] 
			{Вызов минимизатора методом SLSQP};
			\node (check) [proc, below=of solve] 
			{Проверка успешности решения};
			\node (out) [proc, below=of check] 
			{Форматирование вывода};
			
			\draw[con] (func) -- (init);
			\draw[con] (init) -- (constr);
			\draw[con] (constr) -- (solve);
			\draw[con] (solve) -- (check);
			\draw[con] (check) -- (out);
		\end{tikzpicture}
	\end{center}
	
	% Блок-схема 3: transform_fractional_problem()
	\newpage
	\begin{center}
		\begin{tikzpicture}[node distance=1.5cm]
			\node (func) [rec] {transform\_fractional\_problem()};
			\node (vars) [proc, below=of func] 
			{Создание расширенной системы переменных $[y_1,...,y_4, t]$};
			\node (matrix) [proc, below=of vars] 
			{Построение матриц ограничений};
			\node (solve) [proc, below=of matrix] 
			{Решение задачи ЛП методом HiGHS};
			\node (recover) [proc, below=of solve] 
			{Восстановление решения $x_i = y_i/t$};
			\node (check) [proc, below=of recover] 
			{Проверка условия $t > 1\cdot10^{-8}$};
			
			\draw[con] (func) -- (vars);
			\draw[con] (vars) -- (matrix);
			\draw[con] (matrix) -- (solve);
			\draw[con] (solve) -- (recover);
			\draw[con] (recover) -- (check);
		\end{tikzpicture}
	\end{center}
	
	\newpage
	
	
	
	
	\section{Код программы:}
	\begin{code}
import numpy as np
from scipy.optimize import linprog


def transform_fractional_problem():
	"""Преобразование дробной задачи в линейную форму"""
	# Коэффициенты целевой функции и ограничений
	numerator_coeff = [9, 3, 2, 1]  # Числитель
	denominator_coeff = [2, 1, 3, 1]  # Знаменатель

	# Матрица ограничений исходной задачи
	A_orig = [
		[6, 1, 1, 3],
		[1, 0, 3, 2],
		[2, 4, 1, 0]
	]
	b_orig = [280, 70, 320]

	# Создание расширенной системы переменных [y1, y2, y3, y4, t]
	c = numerator_coeff + [0]  # Целевая функция: 9y1 + 3y2 + 2y3 + y4

	# Матрица ограничений для преобразованной задачи
	A_eq = []
	b_eq = []

	# Добавление ограничений из исходной задачи
	for i in range(len(A_orig)):
		row = A_orig[i] + [-b_orig[i]]
		A_eq.append(row)
		b_eq.append(0)

	# Ограничение знаменателя: 2y1 + y2 + 3y3 + y4 = 1
	denom_row = denominator_coeff + [-1]
	A_eq.append(denom_row)
	b_eq.append(1)

	# Решение расширенной задачи ЛП
	res = linprog(
		c=c,
		A_eq=A_eq,
		b_eq=b_eq,
		bounds=[(0, None)] * 5,
		method='highs'
	)

	if res.success:
		t = res.x[-1]
		if t > 1e-8:
			solution = [res.x[i] / t for i in range(4)]
			z_value = sum(numerator_coeff[i] * solution[i] for i in range(4)) / sum(
				denominator_coeff[i] * solution[i] for i in range(4))

			print("Оптимальное решение:")
			print(f"x₁ = {solution[0]:.2f}")
			print(f"x₂ = {solution[1]:.2f}")
			print(f"x₃ = {solution[2]:.2f}")
			print(f"x₄ = {solution[3]:.2f}")
			print(f"Максимальное z = {z_value:.4f}")
		else:
			print("Решение не соответствует условиям")
	else:
		print("Оптимальное решение не найдено")


if __name__ == "__main__":
	transform_fractional_problem()
	\end{code}
	
	Результат работы программы:
	
	\drawzalupa{images/task1.png}{0.6}
	
	
	\newpage
	
	\chapter{Аналитическое решение}
\subsection*{1. Постановка задачи}
Требуется максимизировать целевую функцию:
\[
z = \frac{9x_1 + 3x_2 + 2x_3 + x_4}{2x_1 + x_2 + 3x_3 + x_4} \rightarrow \max
\]
при ограничениях:
\[
\begin{cases}
	6x_1 + x_2 + x_3 + 3x_4 \leq 280, \\
	x_1 + 3x_3 + 2x_4 \leq 70, \\
	2x_1 + 4x_2 + x_3 \leq 320, \\
	x_i \geq 0 \quad (i = 1, 2, 3, 4).
\end{cases}
\]

\subsection*{2. Преобразование задачи ДЛП в ЛП}
Введем замену переменных:
\[
y_0 = \frac{1}{2x_1 + x_2 + 3x_3 + x_4}, \quad y_i = y_0 x_i \quad (i = 1, 2, 3, 4).
\]
Целевая функция преобразуется к виду:
\[
z = 9y_1 + 3y_2 + 2y_3 + y_4 \rightarrow \max.
\]
Ограничения перепишем через новые переменные:
\[
\begin{cases}
	6y_1 + y_2 + y_3 + 3y_4 \leq 280y_0, \\
	y_1 + 3y_3 + 2y_4 \leq 70y_0, \\
	2y_1 + 4y_2 + y_3 \leq 320y_0, \\
	2y_1 + y_2 + 3y_3 + y_4 = 1, \\
	y_i \geq 0 \quad (i = 0, 1, 2, 3, 4).
\end{cases}
\]

\subsection*{3. Решение задачи ЛП симплекс-методом}
Для решения введем искусственную переменную $u$ в уравнение $2y_1 + y_2 + 3y_3 + y_4 + u = 1$. Целевая функция с учётом штрафа $M$:
\[
z_1 = 9y_1 + 3y_2 + 2y_3 + y_4 - M u \rightarrow \max.
\]

\begin{center}
	\textbf{Начальная симплекс-таблица:}
	\vspace{0.5cm}
	
	\begin{tabular}{|c|c|cccccc|}
		\hline
		Базис & Св.чл. & $y_1$ & $y_2$ & $y_3$ & $y_4$ & $y_0$ & $u$ \\ \hline
		$y_0$ & 0 & 6 & 1 & 1 & 3 & -280 & 0 \\
		$y_0$ & 0 & 1 & 0 & 3 & 2 & -70 & 0 \\
		$y_0$ & 0 & 2 & 4 & 1 & 0 & -320 & 0 \\
		$u$ & 1 & 2 & 1 & 3 & 1 & 0 & 1 \\ \hline
		$z_1$ & -M & $9-2M$ & $3-M$ & $2-3M$ & $1-M$ & 0 & 0 \\ \hline
	\end{tabular}
\end{center}

\vspace{0.5cm}
\textbf{Итерация 1.} Разрешающий столбец $y_1$, строка $u$. Новый базис: $y_1$, $y_0$, $y_0$, $y_0$.

\vspace{0.5cm}
\textbf{Итерация 2.} Разрешающий столбец $y_2$, строка $y_0$. После преобразований получаем оптимальную таблицу:

\begin{center}
	\begin{tabular}{|c|c|ccccc|}
		\hline
		Базис & Св.чл. & $y_1$ & $y_2$ & $y_3$ & $y_4$ & $y_0$ \\ \hline
		$y_0$ & 0.0061 & 0 & 0 & 0 & 0 & 1 \\
		$y_1$ & 0.211 & 1 & 0 & 0.071 & 0 & 0 \\
		$y_2$ & 0.363 & 0 & 1 & 0.214 & 0 & 0 \\
		$y_3$ & 0.071 & 0 & 0 & 1 & 0 & 0 \\ \hline
		$z_1$ & 3.1342 & 0 & 0 & 0 & 0 & 0 \\ \hline
	\end{tabular}
\end{center}

\subsection*{4. Возврат к исходным переменным}
\[
x_1 = \frac{y_1}{y_0} = \frac{0.211}{0.0061} \approx 34.76, \quad 
x_2 = \frac{y_2}{y_0} \approx 59.68, \quad 
x_3 = \frac{y_3}{y_0} \approx 11.75, \quad 
x_4 = 0.
\]
Максимальное значение целевой функции:
\[
z_{\text{max}} = 3.1342.
\]

\subsection*{Ответ:}
\[
x_1 = 34.76,\quad x_2 = 59.68,\quad x_3 = 11.75,\quad x_4 = 0.00,\quad z_{\text{max}} = 3.1342.
\]

	\textbf{Вывод: }результат работы программы совпадает с результатом аналитического решения,  значит выполнение программы и аналитического решения дает верные ответы.
\end{document}

